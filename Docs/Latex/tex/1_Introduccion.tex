\capitulo{1}{Introducción}

SpotMyFM es una plataforma diseñada para gestionar la biblioteca y playlists personales de un usuario de Spotify. La principal motivación para enfocar el proyecto alrededor de Spotify es que es una plataforma es muy conservadora a la hora gestionar playlists, bibliotecas y recomendaciones, ya que tienen como filosofía mantener al usuario en un entorno lo más familiar posible.\\
Este comportamiento se ve reflejado en servicios como Discovery Weekly\footnote{Descubrimiento Semanal}, Mix o incluso el modo de reproducción aleatorio ya que tienden a favorecer canciones que has escuchado recientemente, evitando mostrar un gran número de canciones o artistas diferentes. \\
Esta filosofía ha sido objeto de estudio, como explora \cite{spotify_recommendations}, llegando a la conclusión de que cuanto más familiar es el contenido para el usuario, más horas pasa en la plataforma.

Para compensar esta decisión de diseño y mejorar la experiencia de usuario de Spotify, se ha decidido diseñar e implementar una palataforma que permita gestionar la biblioteca personal del usuario, así como sus playlists, de una manera más avanzada. 

La plataforma web incluye 7 servicios diferenciados:
\begin{enumerate}
    \item \textbf{Gestión de Favoritos:} Es la página principal, muestra las canciones y artistas más escuchados en varios intervalos.
    \item \textbf{Gestión de Biblioteca:} Permite explorar todas las características de la biblioteca personal del usuario\footnote{Las canciones marcadas como "Me Gusta"}
    \item \textbf{Gestión de Álbumes:} Permite gestionar y etiquetar los álbumes favoritos del usuario. El sistema de etiquetar álbumes se asemeja a la creación de listas de álbumes. 
    \item \textbf{Gestión de Playlists:} Permite explorar las playlists creadas o marcadas como me gusta por el usuario. Se pueden detallar cada una de las playlists.
    \item \textbf{Análisis de Canción:} Permite subir y analizar una canción cualquiera. El analizador devuelve los géneros, subgéneros y estados de ánimo de la canción, así como 5 canciones similares. 
    \item \textbf{Búsqueda:} Permite buscar a cualquier artista, canción, álbum o playlist y disfrutar de las características que ofrece la plataforma. 
    \item \textbf{Reproductor:} Permite reproducir álbumes o canciones específicas, así como añadir a la cola, saltar o retroceder una canción.
\end{enumerate}

Cada una de estos servicios incluye una vista con los distintos elementos en formato tarjeta interactiva o listado compacto, que pueden ser detallados. Esta lista puede ser filtrada (mediante texto) u ordenada por algún atributo.

\begin{enumerate}
    \item \textbf{Canciones:} Esta vista permite filtrar canciones, crear o editar playlists y consultar las estadísticas generales y recibir recomendaciones para una selección de canciones.\\
    Las canciones pueden ser detalladas con detalles como los géneros, estados de ánimo, fecha de lanzamiento, canciones similares, etc.
    Por otro lado incluye los detalles de de los artistas y álbumes asociados, como las etiquetas de los álbumes, géneros de los artistas, descripción del álbum o un listado con las propias canciones que incluye el álbum de la canción.
    \item \textbf{Playlists:} Esta vista permite listar todas las playlists del usuario. Cada playlist puede ser detallada junto a sus canciones, con todas las características de la vista anterior. 
    \item \textbf{Álbumes:} Esta vista permite filtrar álbumes, así como etiquetarlos. Los detalles de los álbumes son los mismos que los mencionados en el apartado de las canciones.
    \item \textbf{Artistas:} Esta vista permite listar y detallar a los distintos artistas. Al detallar un artista se pueden explorar sus géneros, así como detallar los álbumes, singles y recopilatorios que ha publicado.
\end{enumerate}

Para acompañar todas estas características se ha usado la API pública de LastFM, y se ha desarrollado Ludwig, un servicio que permite, a partir del un fragmento de una canción, obtener canciones similares en Spotify, géneros, subgéneros y los estado de ánimo de una canción. \\
Este segundo servicio utiliza redes neuronales convolucionales (CNN) para clasificar las canciones a partir de su representación mediante sus coeficientes cepstrales en la frecuencia de mel.

Se ha desplegado el proyecto en \href{https://spotmyfm.jorgeruizdev.com}{spotmyfm.jorgeruizdev.com} con una pequeña demo interactiva sin necesidad de registro con algunas de las funcionalidades. 

\subsection{Estructura de la Memoria}

\begin{enumerate}
    \item \textbf{Introducción}: Descripción del proyecto y estructura de la documentación.
    \item \textbf{Objetivos del Proyecto}: Objetivos generales, personales y técnicos que se esperan cumplir con este proyecto.
    \item \textbf{Conceptos Teóricos}: Explicación detallada sobre los conceptos más importantes del proyecto.
    \item \textbf{Técnicas y Herramientas}: Listado y descripción de bibliotecas, servicios y técnicas utilizados durante el proyecto, así como una breve justificación y explicación de su uso. 
    \item \textbf{Aspectos Relevantes}: Listado de los puntos más interesantes e importantes que han surgido durante el desarrollo del proyecto. 
    \item \textbf{Trabajos Relacionados}: Exposición de trabajos y plataformas relacionadas con SpotMyFM. 
    \item \textbf{Conclusiones y Lineas de Trabajo Futuras}: Conclusiones sobre los distintos aparatados del proyecto y un planteamiento de las mejoras que puede recibir el proyecto a lo largo del tiempo.
\end{enumerate}



\subsection{Estructura de los Anexos}

\begin{enumerate}[A.]
    \item \textbf{Plan del proyecto}: Plan de proyecto software que estudia la planificación y viabilidad del proyecto. 
    \item \textbf{Especificación de Requisitos}: Catálogo de los requisitos funcionales y no funcionales del proyecto.
    \item \textbf{Especificación de Diseño}: Diseño general de los distintos apartado del sistema así como su justificación.
    \item \textbf{Manual de Programador}: Manual con toda la información importante que permita a otro programado retomar el proyecto rápidamente.
    \item \textbf{Manual de Usuario}: Manual de usuario que explora todas las características de la plataforma web. 
\end{enumerate}

\subsection{Otros Materiales}

\begin{enumerate}
    \item \textbf{Datasets}: Contiene los distintos conjuntos de datos utilizados para los sistemas de clasificación y recomendación.
    \item \textbf{NextJS}: Contiene la aplicación de Frontend y Backend Principal.
    \item \textbf{Ludwig}: Contiene las herramientas, notebooks y backend utilizados para el servicio de recuperación de la información musical y recomendación.
    \item \textbf{Docker}: Contiene un fichero compose.yml que permite levantar todos los servicios automáticamente.
\end{enumerate}