\apendice{Especificación de Requisitos}

\section{Introducción}

\section{Objetivos generales}

\section{Catalogo de requisitos}

\subsection{Requisitos funcionales}\label{requisitos-funcionales}
\begin{itemize}
\tightlist

    \item
        \textbf{RF-1 Gestión de Biblioteca:} Se requiere que la web sea capaz de mostrar la biblioteca de música personal del usuario.
        \begin{itemize}
           \tightlist
           
            \item
                \textbf{RF-1.1 Listar Biblioteca:} Se requiere que el usuario pueda visualizar su biblioteca de usuario
                \begin{itemize}
                    \item
                        \textbf{RF-1.1.1 Filtrar Vista:} Se requiere que el usuario pueda filtrar la vista de su biblioteca mediante filtros avanzados.
                    \item
                        \textbf{RF-1.1.2 Seleccionar Vista:} Se requiere que el usuario pueda marcar como seleccionados los elementos filtrados.
                    \item
                        \textbf{RF-1.1.3 Ordenar Vista:} Se requiere que el usuario pueda ordenar la vista actual a partir de parámetros.
                    \item
                        \textbf{RF-1.1.3 Detallar Canción:} Se requiere que el usuario pueda ver detalles de cualquier item de la vista. 
                \end{itemize}
                
            \item
                \textbf{RF-1.2 Crear Playlist:} Se requiere que el usuario pueda crear playlists a partir de una selección de canciones.   
                \begin{itemize}
                    \item 
                        \textbf{RF-1.2.1 Configurar Playlist:} Se requiere que el usuario pueda escoger el título, descripción y opciones de privacidad de cada playlist.
                    \item 
                        \textbf{RF-1.2.2 Dividir Playlist:} Se requiere que el usuario pueda dividir una playlist en múltiples playlists.
                \end{itemize}
                
            \item
                \textbf{RF-1.3 Ampliar Playlists:} Se requiere que el usuario pueda ampliar sus playlists ya creadas a partir de una selección de canciones.
                \begin{itemize}
                    \item
                        \textbf{RF-1.3.1 Buscar Playlist:} Se requiere que el usuario pueda seleccionar su playlist a partir de un listado.
                    \item
                        \textbf{RF-1.3.2 Detallar Playlist:} Se requiere que el usuario pueda ver los detalles de una playlist.
                \end{itemize}
                
            \item
                \textbf{RF-1.4 Detallar Playlist:} Se requiere que el usuario pueda ver los detalles de su playlist.
        \end{itemize}
    

        
    \item  
        \textbf{RF-2 Gestión de Álbumes:} Se requiere que la web sea capaz de mostrar los álbumes del usuario.
        \begin{itemize}
            \item 
                \textbf{RF-2.1 Listar Álbumes:} Se requiere que el usuario pueda visualizar sus álbumes.
                \begin{itemize}
                    \item 
                        \textbf{RF-2.1.1 Álbumes Favoritos:} Se requiere que el usuario pueda visualizar sus álbumes marcados como favoritos.
                    \item 
                        \textbf{RF-2.1.2 Álbumes Etiquetados:} Se requiere que el usuario pueda visualizar sus álbumes etiquetados.
                        
                    \item
                        \textbf{RF-2.1.3 Filtrar Vista:} Se requiere que el usuario pueda filtrar la vista actual de sus álbumes mediante filtros avanzados.
                    \item
                        \textbf{RF-1.1.3 Ordenar Vista:} Se requiere que el usuario pueda ordenar la vista actual a partir de parámetros.
                    \item
                        \textbf{RF-1.1.3 Detallar Álbum:} Se requiere que el usuario pueda ver detalles de cualquier álbum en la vista. 
                \end{itemize}
                
            \item 
                \textbf{RF-2.2 Buscar Álbumes:} Se requiere que el usuario pueda visualizar álbumes a partir de una cadena.
            \item
                \textbf{RF-2.3 Gestionar Álbumes Favoritos:} Se requiere que el usuario pueda marcar o desmarcar cualquier álbum como favorito.
            \item
                \textbf{RF-2.4 Etiquetar Álbumes:} Se requiere que el usuario Añadir etiquetas a cualquier álbum.
                \begin{itemize}
                    \item 
                    \textbf{RF-2.4.1 Etiquetas Personalizadas:} Se requiere que el usuario pueda crear sus propias etiquetas.
                \end{itemize}
        \end{itemize}

    \item
        \textbf{RF-3 Detallar Elementos:} Se requiere que el usuario pueda ver detalles de un álbum, playlist o seleccionado por el usuario.
            \begin{itemize}
                \item 
                \textbf{RF-3.1 Detallar Canción:} Se requiere que el usuario pueda ver detalles de una canción específica.
                    \begin{itemize}
                        \item 
                            \textbf{RF-3.1.1 Previsualizar Canción:} Se requiere que el usuario pueda escuchar un fragmento de la canción.
                        \item
                            \textbf{RF-3.1.2 Analizar Canción:} Se requiere que el usuario pueda obtener detalles especiales a partir de un análisis de un fragmento de la canción.
                        \item
                            \textbf{RF-3.1.3 Reproducir Canción:} Se requiere que el usuario pueda reproducir o añadir a la cola una canción en un cliente de Spotify.
                        \item
                            \textbf{RF-3.1.4 Detallar Álbum:} Se requiere que el usuario obtenga los detalles del álbum al que pertenece dicha canción.

                    \end{itemize}
                    
                \item 
                \textbf{RF-3.2 Detallar Álbum:} Se requiere que el usuario pueda ver detalles de un álbum específico.
                    \begin{itemize}
                        \item 
                            \textbf{RF-3.2.1 Gestionar Etiquetas:} Se requiere que el usuario pueda visualizar o gestionar las etiquetas de un álbum
                        \item 
                            \textbf{RF-3.2.2 Conocer Estadísticas:} Se requiere que el usuario pueda ver las estadísticas del álbum mediante LastFM.
                        \item 
                            \textbf{RF-3.2.3 Reproducir Álbum:} Se requiere que el usuario pueda reproducir el álbum en un cliente Spotify.                 
                        \item 
                            \textbf{RF-3.2.3 Detallar Artistas:} Se requiere que el usuario pueda conocer los detalles de los artistas que han participado en el álbum.
                    \end{itemize}
                
                \item
                    \textbf{RF-3.3 Detallar Artistas:} Se requiere que el usuario pueda conocer los detalles de un artista en específico.
                    \begin{itemize}
                        \item \textbf{RF-3.3.1 Géneros Musicales} Se requiere que el usuario pueda conocer los géneros musicales de un artista. 
                    \end{itemize}
            \end{itemize}

    \item
        \textbf{RF-4 Gestión de Tema:} Se requiere que el usuario pueda cambiar el tema actual en cualquier punto de la web.
            \begin{itemize}
                \item 
                \textbf{RF-4.1 Tema Persistente:} Se requiere que el tema seleccionado se mantenga entre sesiones. 
            \end{itemize}
            
    \item
        \textbf{RF-5 Gestión de Sesión:} Se requiere que el usuario tenga el control de la sesión actual.
        \begin{itemize}
            \item
                \textbf{RF-5.1 Cerrar Sesión:} Se requiere que el usuario pueda cerrar su sesión desde la web.
                \begin{itemize}
                    \item
                    \textbf{RF-5.1.1 Limpieza de Sesión:} Se requiere que todos los datos locales sean borrados al cerrar sesión.
                \end{itemize}
            \item
                \textbf{RF-5.2 Limpiar Datos Locales:} Se requiere que los datos locales puedan borrarse desde la web.
        \end{itemize}
    
    \item
        \textbf{RF-5 Estadísticas:} Se requiere que el usuario pueda visualizar las estadísticas relacionadas con su cuenta.
        \begin{itemize}
            \item \textbf{RF-5.1 Listar Canciones}: Se requiere que el usuario pueda conocer sus canciones más escuchadas en diversos periodos de tiempo.
            \item \textbf{RF-5.2 Listar Artistas}: Se requiere que el usuario pueda conocer sus artistas más escuchadas en diversos periodos de tiempo.
        \end{itemize}
    
    \item
        \textbf{RF-6 Cachear Peticiones:} Se requiere que la aplicación realice el mínimo número de peticiones a la API para no sobrepasar los límites.
            \begin{itemize}
                \item \textbf{RF-6.1 Cachear Canciones:} Se requiere que los datos de las canciones se almacenen de forma local.
                \item \textbf{RF-6.2 Cachear Artistas:} Se requiere que los datos de los artistas se almacenen de forma local.
                \item \textbf{RF-6.3 Cachear Álbumes:} Se requiere que los datos de los álbumes se almacenen de forma local.
                \item \textbf{RF-6.4 Refrescar Caché:} Se requiere que los distintos datos almacenados puedan actualizarse para evitar inconsistencias.
            \end{itemize}
    \item
        \texbf{RF-7 Analizar Canciones:} Se requiere que el usuario pueda conocer detalles de cada una de las canciones a partir de un fragmento de la canción.
    
    
\end{itemize}

\subsection{Requisitos no funcionales}\label{requisitos-no-funcionales}
\begin{itemize}
    \item \textbf{RNF-1 Diseño Responsive:} Se requiere un diseño "responsive" para poder utilizarla en dispositivos con distintos tamaños de pantalla o relaciones de aspecto sin perder información.
    \item \textbf{RNF-2 Minimizar Peticiones:} Se requiere minimizar el número de peticiones posibles a las  distintas APIs para evitar alcanzar los límites ver \textbf{RF-6}). 
    \item \textbf{RNF-3 Internacionalización:} Se requiere que la web esté disponible en, al menos, dos idiomas. Algunos elementos como las fechas de lanzamiento deberán tener el formato de fecha adecuado. 
    
    \item\textbf{RNF-4 Compatibilidad:} Se requiere que la web sea funcional a lo largo de los motores webs más utilizados (Chromium, Firefox y Apple Webkit).
    \item\textbf{RNF-5 Carga Diferida:} Se requiere que la web evite cargar un exceso de datos si el usuario no va a necesitarlos, por ejemplo aplicando el patón de Carga Diferida para paginar la carga de imágenes.
    \item\textbf{RNF-6 Accesibilidad:} Se requiere que la web cumpla con el mayor número de recomendaciones posible de Web Content Accessibility Guidelines (WCAG) 2.1
    \item\textbf{RNF-7 Rendimiento y Buenas Prácticas: Se requiere que la web funcione correctamente en dispositivos móviles y en escritorio, minimizando el tamaño de la aplicación para acelerar las cargas de JavaScript y evitar las esperas en el navegador.}
    \item\textbf{RNF-8 Seo:} Se requiere que la web pueda ser indexada en distintos motores de búsqueda y si apliquen técnicas para mejorar la posición de esta.
    \item\textbf{RNF-9 Seguridad: Se requiere que la web sea segura, utilizando TSL o SSL y con actualizaciones constantes para evitar paquetes de terceros con vulnerabilidades.}
    \item\textbf{RNF-10 Privacidad: Se requiere almacenar el mínimo de información que puede identificar a un usuario para mejorar la privacidad de los datos.}
\end{itemize}





\section{Especificación de requisitos}

\subsection{Diagrama de Casos de Uso}

\subsection{Casos de Uso}


\begin{table}[]
\centering
\begin{tabular}{r|p{0.6\textwidth}}
\hline
\textbf{CU-01}         & \textbf{Gestionar Tema}                                 \\ \hline
\textbf{Versión}       & 1.0                                                     \\
\textbf{Autor}         & Jorge Ruiz Gómez                                        \\
\textbf{Requisitos}    & RF-4                                         \\
\textbf{Descripción}   & Permite al usuario cambiar el tema general de la página \\ \hline
\textbf{Precondición}  & Ninguna                                                 \\
\textbf{Acciones}      &    \begin{itemize}
                                \item El Usuario entra en la página.
                                \item La web carga el tema almacenado.
                                \item El usuario pulsa el botón de cambio de tema.
                                \item El Tema se alterna.
                                \item La selección del tema se almacena de forma local.
                            \end{itemize}\\
                                                                          
\textbf{Postcondición} & El tema ha cambiado                                     \\
\textbf{Excepciones}   & Ninguna                                                 \\
\textbf{Importancia}   & Baja                                                    \\ \hline
\end{tabular}
\caption{CU-01}
\label{tab:my-table}
\end{table}


\begin{table}[]
\centering
\begin{tabular}{r|p{0.6\textwidth}}
\hline
\textbf{CU-02}         & \textbf{Gestionar Tema}                                 \\ \hline
\textbf{Versión}       & 1.0                                                     \\
\textbf{Autor}         & Jorge Ruiz Gómez                                        \\
\textbf{Requisitos}    & RF-4                                         \\
\textbf{Descripción}   & Permite al usuario cambiar el tema general de la página \\ \hline
\textbf{Precondición}  & La página ha cargado                                                 \\
\textbf{Acciones}      &    \begin{itemize}
                                \item El Usuario entra en la página.
                                \item La web carga el tema almacenado o selecciona un por defecto.
                                \item El usuario pulsa el botón de cambio de tema.
                                \item El Tema se alterna.
                                \item La selección del tema se almacena de forma local.
                            \end{itemize}\\
                                                                          
\textbf{Postcondición} & El tema ha cambiado                                     \\
\textbf{Excepciones}   & Ninguna                                                 \\
\textbf{Importancia}   & Baja                                                    \\ \hline
\end{tabular}
\caption{CU-02}
\label{tab:my-table}
\end{table}









