\capitulo{2}{Objetivos del proyecto}

Este apartado explica de forma precisa y concisa cuales son los objetivos que se persiguen con la realización del proyecto. Se puede distinguir entre los objetivos marcados por los requisitos del software a construir y los objetivos de carácter técnico que plantea a la hora de llevar a la práctica el proyecto.

\section{Objetivos Generales}
\begin{enumerate}
    \item
        Diseñar e implementar un servicio web que extienda la capacidades de un servicio ya existente.
    \item 
        Diseñar e implementar una arquitectura que interconecte múltiples servicios
    \item
        Diseñar e implementar un servicio que permita analizar y recomendar canciones a partir de un fichero de audio.
        
    
\end{enumerate}

\section{Objetivos Personales}
\begin{enumerate}
    \item
        Estudiar el diseño de una API comercial e integrarla en un proyecto.
    \item
        Conocer el ciclo de desarrollo completo\footnote{Análisis, especificación, diseño, desarrollo, pruebas, SEO y despliegue} de un proyecto web.
    \item
        Implementar, entrenar e integrar un modelo de deeplearning  con una aplicación.
    \item
        Generar un dataset robusto y variado. 
    \item
        Ampliar conocimientos de HTML5 y CSS3 a partir del diseño e implementación de componentes web. 
\end{enumerate}

\section{Objetivos Técnicos}
\begin{enumerate}
    \item 
        Desarrollar una aplicación web totalmente responsive\footnote{Los contenidos se adaptan al dispositivo}.
    \item
        Desarrollar una aplicación web con modo claro y oscuro.
    \item
        Poner en práctica los conocimientos de CI/CD \footnote{Integración Continua / Despliegue Continuo}
    \item
        Crear un sistema de recomendación de música colaborativo y basado en contenidos integrado con Spotify.
    \item
        Conocer frameworks y bibliotecas de pruebas, unitarias, automáticas y End to End (E2E).
    \item
        Conocer los distintos conceptos y técnicas para trabajar con audio digital.
\end{enumerate}